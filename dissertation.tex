%% There are no requirements on font size, line spacing and stuff.
%% KOMA uses type area. General guidelines on good readability aim for:
%%      10 to 12 Words/Line, ca 60-70 Characters / Line
%% This should be the default of KOMA's typearea with DIV=calc.
%% BCOR is the binding correction. Update BCOR for final binding!
%%
%% Further readings:
%% documentation -> lb2-ch4.pdf
%% http://www.webmasterpro.de/design/article/typografie-12-wichtige-grundlagen-fuer-den-richtigen-einsatz-von-schriften.html
\documentclass[
    xetex,
    a4paper,fleqn,12pt,
    % twoside,openright,         %% two sided (book), open chapters on the right hand side
    % twocolumn,
    BCOR10mm, DIV=calc,        %% 10mm for binding, keep around 12 words/line
    % DIV=classic,
    % titlepage,
    % draft
    bibtotoc,
    listof=totoc,
    titlepage=firstiscover,
    parskip=half
]{scrbook}

%%%%%%%%%%%%%%%%%%%%%%%%%%%%%%%%%%%%%%%%%%%%%   Language Selection
% Language selection for XeLaTeX (replaces babel)
\usepackage{polyglossia}
\setdefaultlanguage{british}


%%%%%%%%%%%%%%%%%%%%%%%%%%%%%%%%%%%%%%%%%%%%%   General Configuration
% This is a conglomerate of various tex files I (Jan-Matthias Braun) merged
% for my thesis with excerpts from Jeremie Papon's Thesis (https://github.com/jpapon/papon_thesis)
% which itself was forked of the original Harvard Latex template by Jordan Suchow (suchow@fas.harvard.edu).
%
% In case of questions, please contact Jan-Matthias Braun (jan_braun@gmx.net).
% Otherwise feel free to use this template for your own thesis.
%
% Notes
% (1) This file is intended for use with xelatex, but the corresponding
%     lines for pdflatex are included as comments.
% (2) Please consult the comments on Layout-Tuning below.
% (3) This is the future, and it uses UTF-8, you can directly include
%     characters like ℝ, with the right definitions (see UTF-8 below).
%
% TODO: 
% (1) This template still uses the old subfigure package.
%     Please feel inclined to use the subfig package and
%     send me the patches. :-)
% (2) The title page has too many things hard coded.
%     You might have to touch the code.
% (3) There are lots of other things to do, please fork
%     and prepare your changes for public access or send
%     them back to me.

%%%%%%%%%%%%%%%%%%%%%%%%%%%%%%%%%%%%%%%%%%%%%%%%%%%%%%%%%%%%%%%%%   General Packages

\usepackage{fixltx2e}                 % Obsolete with recent LaTeX distributions
\usepackage{microtype}                % defaults are protrusion=true,expansion=true

\usepackage{setspace}                 % line-spacing
\onehalfspace                         % \onehalfspace corresponds to \setstretch{1.2}
% \doublespace                        % To have more space for corrections.

\usepackage{amsfonts}                 % math packages
\usepackage{amssymb}
\usepackage{amsmath}

\usepackage{graphicx}
\usepackage[vfloat]{floatflt}
\usepackage{float}

% sidewaysfigure and rotations
\usepackage[figuresright]{rotating}

% Change the style of pages with huge floats with \thisfloatpagestyle
% http://tex.stackexchange.com/questions/169908/how-to-set-the-pagestyle-on-the-page-a-particular-float-ends-up-on
\usepackage{floatpag}

\usepackage{pdflscape}                                   % Set landscape mode for singular pages

\usepackage[numbers, comma, sort&compress]{natbib}       % Plain bibliography style

\usepackage[dvipsnames]{xcolor}
\usepackage{booktabs}
\usepackage[final]{pdfpages}
\usepackage{multirow}
\usepackage{colortbl}
\usepackage{paralist}
\usepackage{units}


%%%%%%%%%%%%%%%%%%%%%%%%%%%%%%%%%%%%%%%%%%%%%%%%%%%%%%%%%%%%%%%%%%   Abbreviations
% Handle abbreviations properly
% This is useful for inline enumerations, e.g., an explicit 1. ...
\usepackage{xpunctuate}                                  
\newcommand{\ie}{i\xperiod{}e\xperiod{}}                 % Define shortcuts for often used abbreviations
\newcommand{\eg}{e\xperiod{}g\xperiod{}}
\newcommand{\etc}{etc\xperiod{}}
\newcommand{\ff}{ff\xperiod{}}


%%%%%%%%%%%%%%%%%%%%%%%%%%%%%%%%%%%%%%%%%%%%%%%%%%%%%%%%%%%%%%%%%   Enable Synctex

% Synctex allows direct switching between positions in the editor and the pdf viewer.
% For details, see 
% https://www.math.tu-berlin.de/fileadmin/i26/download/AG_ModNumDiff/FG_NumMath/seminars/toolseminar/ts5invSear.pdf
% http://tex.stackexchange.com/questions/161797/how-to-configure-emacs-and-auctex-to-perform-forward-and-inverse-search
\RequirePackage{pdfsync}
\synctex=1


%%%%%%%%%%%%%%%%%%%%%%%%%%%%%%%%%%%%%%%%%%%%%%%%%%%%%%%%%%%%%%%%%   Font selection
% http://www.macfreek.nl/memory/Fonts_in_LaTeX


% Old LaTeX-Style font selection
%http://tex.stackexchange.com/questions/66949/command-nobreakspace-unavailable-when-switching-to-t1-encoding-under-xelatex
% Results in:
% Package	Roman Font	Sans-Serif Font	Monospaced Font	Math Font
% default	CM Roman	CM Sans Serif	CM Typewriter	≈ CM Roman
% lmodern	LM Roman	LM Sans Serif	LM Typewriter	
% palatino	Palatino	Helvetica Courier
%% Previously \usepackage{lmodern}
%\usepackage[T1]{fontenc}                                        % For T1 see end of file!
\usepackage{lato}                                               % Sans serif font
\usepackage{palatino}                                           % Palatino serif font

\usepackage{fontspec}
\defaultfontfeatures{Mapping=tex-text}  % For archaic input (e.g. convert -- to en-dash)

%\setmainfont{CMU Serif}                 % Computer Modern Unicode font (Traditional LaTeX fonts)
%\setmonofont{CMU Typewriter Text}
%\setsansfont{CMU Sans Serif}

% I have the impression, that Palatino is better to read on paper ... 
% On screen I prefer something else.
%\setmainfont[Numbers=OldStyle]{TeX Gyre Pagella}      % Palatino as main font
%\linespread{1.07}
\setmainfont[Numbers=OldStyle]{TeX Gyre Termes}        % Times New Roman
\setsansfont[Numbers=OldStyle]{Lato}
\setmonofont{Source Code Pro}

% Further examples for font selection: Adventor in variations
%\setsansfont{\fontspec[Numbers={Proportional,OldStyle}]{TeX Gyre Adventor}}
%\setsansfont{TeX Gyre Adventor:+onum}
%\setsansfont[Numbers=OldStyle]{TeX Gyre Adventor}


%%%%%%%%%%%%%%%%%%%%%%%%%%%%%%%%%%%%%%%%%%%%%%%%%%%%%%%%%%%%%%%%%   Layout-Tuning

\usepackage[bottom]{footmisc}
% For final paragraph spacing/ragged bottom tuning:
%\raggedbottom
%\enlargethispage{\baselineskip}
% + manual linebreaks
% See: http://www.latex-community.org/forum/viewtopic.php?f=47&t=10862


% To enforce placement of floats two options exist, which do not
% produce a misplaced \clearpage. Still the behaviour is different.
% \afterpage delays the float insertion and therefore guarantees
% that the previous page is filled optimally.
% \FloatBarrier will keep a possible change in content better served
% and may leave the previous page partially empty.
\usepackage{placeins}                       % \FloatBarrier
\usepackage{afterpage}                      % \afterpage{\clearpage}

\usepackage[scriptsize,hang,tight,sf,bf,nooneline]{subfigure}
\usepackage[margin=8mm,font=small,                              % Caption tuning
            labelfont=bf,up,normal]{caption}                    % Indent 8mm, size: small, textbf{figure}
\renewcommand{\belowcaptionskip}{12pt}                          % Tuning of caption and figure spacing
\renewcommand{\intextsep}{12pt}
% \renewcommand{\captionfont}{\small\slshape}                   % Font: small italics
% \addtokomafont{caption}{\small}                               % Another way to say: small font

% This one leads to quite a lot huge spaces between paragraphs
% because of excessive page breaks.
%\RequirePackage[small, md, sc]{titlesec}
% You need the right font to have nice small caps, though.
%\setkomafont{sectioning}{\scshape}
\usepackage[palatino]{quotchap}

%\setsansfont[Numbers=OldStyle]{Lato}
%\setsansfont[Numbers=OldStyle]{TeX Gyre Adventor}
%\setkomafont{disposition}{\bfseries\fontspec[Numbers=OldStyle]{TeX Gyre Adventor}}
\setsansfont[Numbers=OldStyle]{Lato}
\setkomafont{disposition}{\bfseries\fontspec[Numbers=OldStyle]{Lato}}


%%%%%%%%%%%%%%%%%%%%%%%%%%%%%%%%%%%%%%%%%%%%%%%%%%%%%%%%%%%%%%%%%   Colours

\definecolor{Crimson}{rgb}{0.6471, 0.1098, 0.1882}
\definecolor{DarkBlue}{rgb}{0.1216, 0.1843, 0.2902}
\definecolor{Silver}{rgb}{0.5490, 0.5451, 0.5882}
\definecolor{darkblue}{rgb}{0.1216, 0.1843, 0.44}


%%%%%%%%%%%%%%%%%%%%%%%%%%%%%%%%%%%%%%%%%%%%%%%%%%%%%%%%%%%%%%%%%   UTF-8 character definitions
%% Compare http://tex.stackexchange.com/questions/29459/declareunicodecharacter-doesnt-work-for-all-characters
%% with respect to DeclareUnicodeCharacter

\usepackage{textcomp}
\usepackage{gensymb}
\usepackage{newunicodechar}

\newunicodechar{ℝ}{\mathbb{R}}
\newunicodechar{…}{\ldots}
\newunicodechar{°}{\degree}


%%%%%%%%%%%%%%%%%%%%%%%%%%%%%%%%%%%%%%%%%%%%%%%%%%%%%%%%%%%%%%%%%   Draft stage helpers

% For a really nice todo package including colours & listings, see
% https://tex.stackexchange.com/questions/9796/how-to-add-todo-notes
\usepackage[colorinlistoftodos,prependcaption,textsize=tiny]{todonotes}

\usepackage{ifthen}
\usepackage{comment}                 % Allows make comments out of environments
\usepackage{etoolbox}                % Easy test for empty arguments

%% https://en.wikibooks.org/wiki/LaTeX/Colors
%% https://en.wikibooks.org/wiki/LaTeX/Macros

%% Variable to control output of title page, additional lists, appendices, etc.
%% Could be made a package with DeclareOption which alters the variable.
\newboolean{fulldocument}
\setboolean{fulldocument}{false}

\newcommand{\onlyinfulldocument}[1]{%       % Suppress output if not in internaldraft mode
  \ifthenelse{\boolean{fulldocument}}{%
    #1
  }{}%
}%

%% Could be made a package with DeclareOption which alters the variable.
\newboolean{internaldraft}
\setboolean{internaldraft}{true}

\newcommand{\onlyindraft}[1]{%       % Suppress output if not in internaldraft mode
  \ifthenelse{\boolean{internaldraft}}{%
    #1
  }{}%
}%

\ifthenelse{\boolean{internaldraft}} {%
  \newenvironment{outline}{%
    \textcolor{MidnightBlue}\bgroup
    \textbf{Section Outline:}\\
  }{ \egroup}%

  \newenvironment{itemoutline}{%
    \textcolor{MidnightBlue}\bgroup
    \textbf{Section Outline:}\\
    \begin{itemize}
  } {%
    \end{itemize} \egroup
  }%

} {%
  \excludecomment{outline}
  \excludecomment{itemoutline}
}%

\newcommand{\sidemark}{\onlyindraft{\marginpar{\rule[-3mm]{2mm}{6mm}}}}
\newcommand{\sideremark}[1]{\onlyindraft{\marginpar{\rule[-7mm]{2mm}{10mm}~\parbox[t]{32mm}{#1}}}}
\newcommand{\sidenote}[1]{\onlyindraft{\marginpar{#1}}}


%\newcommand{\REMARK}[1]{\sideremark{#1}}
\newcommand{\REMARK}[1]{\onlyindraft{\todo[linecolor=green,backgroundcolor=green!25,bordercolor=green]{#1}}}

%% Marker where I left writing
\newcommand{\TODO}[1][]{
  \onlyindraft{%
    \textcolor{red}{\ldots}%
    \todo[linecolor=red,backgroundcolor=red!25,bordercolor=red]{%
      TODO: \ifstrempty{#1}{to be continued}{#1}%
    }%
  }%
}%

%% Marker for additional citations
\newcommand{\NEEDSCITATIONold}[1][]{
  \onlyindraft{%
    \ifstrempty{#1}{%
      \marginpar{\textcolor{blue}{\rule[-1mm]{1mm}{4mm}}} \footnote{\textcolor{blue}{Citation needed}}%
    }{%
      \marginpar{\textcolor{blue}{\rule[-1mm]{1mm}{4mm}}} \footnote{\textcolor{blue}{Citation needed: #1}}%
    }%
  }%
}%

%% Marker for additional citations
\newcommand{\NEEDSCITATION}[1][]{
  \onlyindraft{%
    \todo[linecolor=blue,backgroundcolor=blue!25,bordercolor=blue]{%
      Citation needed\ifstrempty{#1}{}{: #1}%
    }%
  }%
}%


% %%% Draft revision mark at end of every page %%%%%%%%%%%%
%
% The AddEverypageHook is used to register the printrev command.
% An optional parameter allows arbitrary revision tags.
% \printrev is conditioned on the bool internaldraft.
% When internaldraft is false, the marker is not printed.
\usepackage{everypage}
\usepackage[absolute]{textpos}
\usepackage{datetime}
\newcommand\printrev[1]{%
  \onlyindraft{%
    \begin{textblock*}{\paperwidth}(0mm,\paperheight-16mm)%
      \begin{center}%
        \textit{\footnotesize --- Document rendered on \today{} \currenttime{} #1 ---}%
      \end{center}%
    \end{textblock*}%
  }
}
\AddEverypageHook{\printrev{}}


%%%%%%%%%%%%%%%%%%%%%%%%%%%%%%%%%%%%%%%%%%%%%%%%%%%%%%%%%%%%%%%%%   Title page
% Titlepage and below: courtesy of Jeremie Papon 
% with modifications by Jan-Matthias Braun
% For details of forking and changes, please check the header.

% Trying to unstrip the KOMA title page.
% Although this does not seem to work in any way.
\newcommand*{\mytitlemargin}{\dimexpr 0mm\relax}
\renewcommand{\coverpagetopmargin}{\mytitlemargin}
\renewcommand{\coverpageleftmargin}{\mytitlemargin}
\renewcommand{\coverpagerightmargin}{\mytitlemargin}
\renewcommand{\coverpagebottommargin}{2\mytitlemargin}

% some definitions
\def\title#1{\gdef\title{#1}}
\def\author#1{\gdef\author{#1}}
\def\degreeyear#1{\gdef\degreeyear{#1}}
\def\degreemonth#1{\gdef\degreemonth{#1}}
\def\degree#1{\gdef\degree{#1}}
\def\advisor#1{\gdef\advisor{#1}}
\def\department#1{\gdef\department{#1}}
\def\field#1{\gdef\field{#1}}
\def\university#1{\gdef\university{#1}}
\def\universitycity#1{\gdef\universitycity{#1}}
\def\universitystate#1{\gdef\universitystate{#1}}

\renewcommand{\maketitle}{%
  \begin{titlepage}
  %\singlespacing
  %\thispagestyle{empty}
  \begin{flushright}
  %\vspace*{100pt} 
%  \vspace*{20pt} 
  %\vspace*{\fill}
  \Huge \textcolor{DarkBlue}{\textit{\title}} %\vspace{10pt}
  \textcolor{Silver}{\rule{\textwidth}{2.0pt}}   
  ~ \\ %%%
  \normalsize \normalfont \scshape Dissertation \\ 
  \vspace{6pt}
  in order to obtain the doctoral degree\\
  %in Mathematics and Natural Sciences\\ 
  "Doctor rerum naturalium" \\
  %%% \vspace{6pt}
  of the Georg-August-Universität Göttingen \\
  \vspace{12pt} %%%
  \vspace{6pt}
  in the Doctoral program \\
  Theoretical and Computational Neuroscience (PTCN) of\\
  the Georg-August University School of Science (GAUSS) \\
  \vspace{24pt}
  \vspace{4pt} %%%
  submitted by\\
  \author\\
  \vspace{12pt} 
  of Göttingen, Germany\\
  (place of birth)\\
  \vspace{36pt}
  \vspace*{\fill}
  \includegraphics[width=30mm]{figures/UniLogo1} \\
  \vspace{12pt} 
  \university\\ \universitycity, \universitystate\\
  \degreemonth\ \degreeyear \\
  \end{flushright}
  %\vspace*{\fill}
  \end{titlepage}
}

\newcommand{\abstractpage}{
  \newpage
  %\pagenumbering{roman}
  %\setcounter{page}{3}
  \thispagestyle{empty}
%  \pagestyle{fancy}
%  \renewcommand{\headrulewidth}{0.0pt}
  \vspace*{-28mm}
  \begin{center}
  \vspace*{1pt} % \huge
  \LARGE \textcolor{DarkBlue}{\textit{\title}} \\
  %\vspace*{15pt}
  \vspace*{9pt}
  \Large \scshape Abstract \\ \normalfont \rmfamily
  \end{center}
  \singlespacing

Ramble ramble ramble …

\onehalfspace

%%% Local Variables:
%%% ispell-local-dictionary: "en_US"
%%% coding: utf-8
%%% mode: latex
%%% TeX-master: "dissertation"
%%% TeX-engine: xetex
%%% End:

  \newpage
  %\cfoot{\thepage}
}


%%%%%%%%%%%%%%%%%%%%%%%%%%%%%%%%%%%%%%%%%%%%%%%%%%%%%%%%%%%%%%%%%   Hyperref at the end
% Funny options by Mr. Papon
% There is supposed to be a setup command which is 
% better in terms of encoding issues.
\RequirePackage[bookmarks, 
                colorlinks=true, 
                citecolor=Crimson,
                filecolor=black,
                linkcolor=DarkBlue,
                urlcolor=Crimson,
                pdfdisplaydoctitle=true,
                ]{hyperref} 


%% Only for T1 text-encoding
%% Problem: There is no direct translation from Unicode to T1
%\DeclareTextCommandDefault{\nobreakspace}{\leavevmode\nobreak\ } 
             % This should become a class...
\setboolean{internaldraft}{true}    % Notes and Remarks
\setboolean{fulldocument}{true}     % Include title page


%%%%%%%%%%%%%%%%%%%%%%%%%%%%%%%%%%%%%%%%%%%%%   Chapter names in lists
\usepackage{chaptersinindex}

\AddChaptersToIndex{lof}{figure}
\AddChaptersToIndex{lof}{floatingfigure}
%\AddChaptersToIndex{lot}{table}


%%%%%%%%%%%%%%%%%%%%%%%%%%%%%%%%%%%%%%%%%%%%%   PDF-Metadata
% It is assumed best practise to configure hyperref
% with the setup statement. (In the literature...)
\hypersetup{
    pdftoolbar=true,        % show Acrobat’s toolbar?
    pdfmenubar=true,        % show Acrobat’s menu?
    pdffitwindow=false,     % window fit to page when opened
    pdfstartview={Fit},    % fits the page to the window
    pdftitle={Adaptable Orthosis Control with Aritficial Neural Networks},      % title
    pdfauthor={Jan-Matthias Braun},     % author
    pdfsubject={Dissertation},   % subject of the document
    %pdfcreator={Jan-Matthias Braun},   % creator of the document
    %pdfproducer={Jan-Matthias Braun}, % producer of the document
    %pdfkeywords={Green Networking} {Mobile Cloud} {Network Coding} {Energy}, % list of keywords
    pdfnewwindow=true,      % links in new window
    %hidelinks=true,         % for printing: turn link colours off
}


%%%%%%%%%%%%%%%%%%%%%%%%%%%%%%%%%%%%%%%%%%%%%   PDF-Archive Formats
%% For advanced pdf features, it seems advised to use pdfx.
%% It allows to create archive-variants
%% a for archive, x for printing.
% http://tex.stackexchange.com/questions/87912/add-metadata-in-pdf-as-type-pdf-a
% http://ctan.sciserv.eu/macros/latex/contrib/pdfx/pdfx.pdf
% http://www.mathstat.dal.ca/~selinger/pdfa/
%\usepackage[x-1b]{pdfx}

%% For pdfx, the metadata is supplied in a file.
%% This file can be supplied inline via the filecontents package.
%\usepackage{filecontents}
%\begin{filecontents*}{\jobname.xmpdata}
%\Keywords{pdfTeX\sep PDF/X-1a\sep PDF/A-b}
%\Title{Sample LaTeX input file}
%\Author{LaTeX project team}
%\Org{TeX Users Group}
%\Doi{123456789}
%\end{filecontents*}


%%%%%%%%%%%%%%%%%%%%%%%%%%%%%%%%%%%%%%%%%%%%%   Glossaries
\usepackage[acronym,nonumberlist,nomain,toc]{glossaries}
\setacronymstyle{long-sc-short}
\makeglossaries
\loadglsentries{acronyms}


%%%%%%%%%%%%%%%%%%%%%%%%%%%%%%%%%%%%%%%%%%%%%   You need a watermark?
%\onlyindraft{
%  \watermark {DRAFT COPY ONLY}
%}


%%%%%%%%%%%%%%%%%%%%%%%%%%%%%%%%%%%%%%%%%%%%%   Document start
\begin{document}

%% Kein Bedarf an unnötigem Titel-Bla-Bla zum jetztigen Zeitpunkt.
\onlyinfulldocument{
  \title{Your Lenghty Title Here}
\author{That's You}

% about the degree
%\degree{Doctor of Philosophy}
%\field{Physics}
\degreeyear{The Year}
\degreemonth{The Month}

% about the university
\department{Faculty of Natural Sciences and Mathematics}
\university{Georg-August-Universität Göttingen}
\universitycity{Göttingen}
\universitystate{Germany}

\maketitle

\newpage \thispagestyle{empty}
\underline{Thesis committee}
\vskip 2.0mm
\textbf{Prof. No One}, \\
\small Institute One

\vskip 0.05cm
\normalsize \textbf{Prof. No Two}, \\
\small Department Two

\vskip 0.05cm
\normalsize \textbf{Prof. Three}, \\
\small Group Three

\vskip 4.0mm
\normalsize \underline{Members of the examination board}
\vskip 2.0mm
\normalsize First Reviewer: \textbf{Prof. A}, \\
\small Institute for A-Studies

\vskip 0.05cm
\normalsize Second Reviewer: \textbf{Prof. B}, \\
\small Whetever

\vskip 4.0mm
\normalsize Other members of the examination board:
\vskip 2.0mm
\normalsize \textbf{Prof. C}, \\
\small I really don't want to invent stuff anymore

\vskip 0.05cm
\normalsize \textbf{Prof. D}, \\
\small Bli

\vskip 0.05cm
\normalsize \textbf{Prof. E}, \\
\small Bla

\vskip 0.05cm
\normalsize \textbf{Prof. F}, \\
\small Blubb


\normalsize

\vfill
Date of the oral examination: \hfill XX\textsuperscript{th} of Month, Year

\cleardoublepage
%\thispagestyle{empty}
%~\newpage{}
%\thispagestyle{empty}
%\cleardoublepage
\pagenumbering{roman}\setcounter{page}{3}  
\abstractpage
\thispagestyle{empty}
\cleardoublepage

%%% Local Variables:
%%% ispell-local-dictionary: "en_US"
%%% TeX-master: "dissertation"
%%% End:

}

\tableofcontents

\printglossary[type=\acronymtype,
               title=List of Acronyms,
               toctitle=Terms and Abbreviations]

\onlyinfulldocument{      % Use this to remove preambles
  \listoffigures
  \listoftables
}

%%%%%%%%%%%%%%%%%%%%%%%%%%%%%%%%%%%%%%%%%%%%%%%%%%%%%%%%%%%%%%%%%%%%%%%%%%
%%%
%%%              Chapters
%%%
\cleardoublepage
\setcounter{page}{1}
\pagenumbering{arabic}
\onehalfspace

\begin{savequote}[75mm]
Our freedom can be measured by the number of things we can walk away from.
\qauthor{Source unknown (attributed to Vernon Howard)}
\end{savequote}

\chapter{One}
\label{cha:one}

\section{Notes and Acronyms}
\label{sec:notes-acronyms}


\TODO{You can add todo notes everywhere,}
Use acronyms, demonstrate abbreviations... \Gls{one}, \gls{two}, \glspl{one}.

\glsreset{one}
In case you had quite some distance to the last use of an acronym or start a new chapter, you might want to reintroduce it. Use a reset for this and produce another reference to \gls{one}.
\REMARK{The notes are only visible in internaldraft mode. This additional mode exists to disambiguate to latex standard draft mode, which will replace pictures by rectangles. Additionally, a time stamp at the bottom helps to sort through revisions.}

\begin{figure}[!ht]
  \centering
  \includegraphics{figures/UniLogo1.png}
  \caption{Logo of University Göttingen (1)}
  \label{fig:logo-1}
\end{figure}


\section{Abbreviations}
\label{sec:abbreviations}

To cope with the special handling of punctuation, you can help \LaTeX with the xpunctuate package, \ie, prevent extra spaces after 1\xperiod{}~abbreviation dots which are misinterpreted as the end of the sentence \etc. For example in etc. without the extra markup, which might produce extra space when the line is filled. 2.~After inline enumerations, which might break the sentence-structure with the same over-wide spaces otherwise.


\section{Pre-Print Beautification}
\label{sec:pre-print-beaut}

Only do this if you are sure that nothing changes anymore.\NEEDSCITATION[cite something?]

\begin{enumerate}
\item Solve all issues (notes, todos, etc.)
\item Compile without draft notes (otherwise the layout may change).
\item Check the positioning of all floats (e.g., figures and tables).
\item Make sure that there are no overfull hboxes. Indicate appropriate word-break positions or change the sentence.
\item Make sure that there are no single lines or section titles at the end of pages (or single lines at the start...) Use enlargethispage (on both sides of the double page!) to make things fit.
\item Now make sure that the toc looks nice by application of appropriate addtocontents lines.
\item When all pages are in position, do whatever beatifications you like, for example, apply a suitable style to empty pages before chapters.
\end{enumerate}

Please consult the comments in definitions.tex, section Layout-Tuning, for further help to get figures and text aligned nicely before final printing.

% Sometimes a (double) page has to be adjusted to
% prevent ugly paragraph setting.
% Unluckily, there is no enlargethisdoublepage command,
% so you might have to do this twice. ;-)
\enlargethispage{4\baselineskip}


%%% Local Variables:
%%% ispell-local-dictionary: "british"
%%% coding: utf-8
%%% mode: latex
%%% TeX-master: "dissertation"
%%% TeX-engine: xetex
%%% End:

\addtocontents{toc}{\protect\pagebreak[4]}   % Beautify the toc with 
                                             % forced pagebreaks.
%\begin{savequote}[75mm]
%\end{savequote}

\chapter{Two}
\label{cha:two}

In which we have no quotation.

For initial structuring and sketching, there are a few environments which are nicely colour coded and only visible in draft mode.

\begin{outline}
  There is the free-text outline.
\end{outline}

\begin{itemoutline}
\item And of course, for convenience, the itemoutline.
\item Use it as fits when structuring your thesis
\item And later replace the bullet points with contents.
\end{itemoutline}

Use the comment environment to easily comment out larger sections of text.

\begin{figure}[!ht]
  \centering
  \includegraphics{figures/UniLogo1.png}
  \caption{Logo of University Göttingen (2)}
  \label{fig:logo-2}
\end{figure}

\begin{comment}
  Bla bla bla bla bla bla bla bla bla bla bla bla bla bla bla bla bla bla bla bla bla bla bla bla bla bla bla bla bla bla bla bla bla bla bla bla bla bla bla bla bla bla bla bla bla bla bla bla bla bla bla bla bla bla bla bla bla bla bla bla bla bla bla bla bla bla bla bla bla bla bla bla bla bla bla bla bla bla bla bla bla bla bla bla bla bla bla bla bla bla bla bla bla bla bla bla bla bla bla bla bla bla bla bla bla bla bla bla bla bla bla bla bla bla bla bla bla bla bla bla bla bla bla bla bla bla bla bla bla bla bla bla bla bla bla bla bla bla bla bla bla bla bla bla bla bla bla bla bla bla bla bla bla bla bla bla bla bla bla bla bla bla bla bla bla bla bla bla bla bla bla bla bla bla bla bla bla bla bla bla bla bla bla bla bla bla bla bla bla bla bla bla bla bla bla bla bla bla bla bla bla bla bla bla bla bla bla bla bla bla bla bla bla bla bla bla bla bla bla bla bla bla bla bla bla bla bla bla bla bla bla bla bla bla bla bla bla bla bla bla bla bla bla bla bla bla bla bla bla bla bla bla bla bla bla bla bla bla bla bla bla bla bla bla bla bla bla bla bla bla bla bla bla bla bla bla bla bla bla bla bla bla bla bla bla bla bla bla bla bla bla bla bla bla bla bla bla bla bla bla bla bla bla bla bla bla bla bla bla bla bla bla bla bla bla bla bla bla bla bla bla!
\end{comment}
%%% Local Variables:
%%% ispell-local-dictionary: "british"
%%% coding: utf-8
%%% mode: latex
%%% TeX-master: "dissertation"
%%% TeX-engine: xetex
%%% End:



%%%%%%%%%%%%%%%%%%%%%%%%%%%%%%%%%%%%%%%%%%%%%%%%%%%%%%%%%%%%%%%%%%%%%%%%%%
%%%
%%%              Bibliography
%%%
%\bibliographystyle{apalike}
%\addcontentsline{toc}{chapter}{References}
\bibliographystyle{plainnat}
\bibliography{bibliography.bib}


%%%%%%%%%%%%%%%%%%%%%%%%%%%%%%%%%%%%%%%%%%%%%%%%%%%%%%%%%%%%%%%%%%%%%%%%%%
%%%
%%%              Appendices
%%%
\appendix
%\begin{savequote}
%\end{savequote}
\chapter*{\vspace*{5cm}Appendices}
% \refstepcounter{chapter}
%\section{Source Code of Foo}
\ldots

%%% Local Variables:
%%% ispell-local-dictionary: "en_US"
%%% coding: utf-8
%%% mode: latex
%%% TeX-master: "dissertation"
%%% TeX-engine: xetex
%%% End:


\onlyindraft{
  \listoftodos[Notes]
}

\end{document}

%%% Local Variables:
%%% ispell-local-dictionary: "british"
%%% coding: utf-8
%%% mode: latex
%%% TeX-master: "dissertation"
%%% TeX-engine: xetex
%%% End:
